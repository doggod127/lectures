\documentclass{article}

\usepackage[utf8]{inputenc}
\usepackage[russian]{babel}

\usepackage{graphicx}
\graphicspath{ {./pics/} }

\title{Вопросы к экзамену по высшей математике для первого семестра первого курса}
\date{14 января 2023 г.}

\begin{document}
	\maketitle
	\newpage
	
	\section{Комплексные числа}
	\subsection{Вопрос 1}
	Сформулируйте определение комплексного числа (КЧ) в алгебраической форме и перечислите его составляющие. \\
	Как соотносятся между собой множества действительных чисел, комплексных, мнимых? \\
	Какие числа называются комплексно-сопряженными? Перечислите основные свойства операций с ними. \\
	Как выполняются операции сложения, вычитания, умножения, деления КЧ в алгебраической форме? \\
	
	\paragraph{}Комплексным число в алгебраической форме называется выражение вида \( z = x + iy\), где $x$ - действительная часть, $y$ - мнимая часть, $i$ - мнимая единица.
	
	\paragraph{} Между собой множества действительных, комплексных и мнимых числе соотносятся так: множество действительных чисел является подмножеством комплексных, так как любое комплексное число можно представить как \( z = x + 0i\); мнимое число является подмножеством комплексных чисел, так как любое мнимое число можно представить как \( m = 0 + xi\). \textbf{TBD}
	
	\paragraph{} Комлексно-сопряженными числами называются те комплексные числа, которые различаются только знаком между частями. Например: \( z_1 = 3 + 4i\), \( z_2 = 3 - 4i\). Основные свойства операций над ними(\( z = x + yi\)):
	\begin {enumerate}
		\item \(
			z + \overline{z} = 2x
		\)
		\item \( 
			z \overline{z} = x^2 + y^2
		\)
		\item \( 
			\overline{z} = z 
		\)
		\item \( 
			\overline{z_1 \pm z_2} = \overline{z_1} \pm \overline{z_2} 
		\)
		\item \( 
			\overline{z_1 z_2} = \overline{z_1} * \overline{z_2} 
		\)
		\item \( 
			\overline{(z_1 \over z_2)} = { {\overline{z_1}} \over \overline{z_2} } 
		\)
	\end {enumerate}
	
	\paragraph{} Пусть \( z_1 = x_1 + y_1i \), \( z_2 = x_2 + y_2i \). Основные операции над комлексными числами:
	\begin{enumerate}
		\item Сложение: \( 
			z_1 + z_2 = x_1 + x_2 + (y_1 + y_2)i 
			\)
		\item Вычитание: \( 
			z_1 - z_2 = x_1 - x_2 + (y_1 - y_2)i 
			\)
		\item Умножение: \( 
			z_1 z_2 = x_1 x_2 - y_1 y_2 + (x_1 y_2 + x_2 y_1)i
			\). 
			Формула получается из раскрытия скобок и приведения подобных.
		\item Деление: \(
			{z_1 \over z_2} = {{z_1 \overline{z_2}} \over {z_2 \overline{z_2}}} 
		\). Формула получается домножением числителя и знаменателя на сопраженное знаменателю число.
	\end{enumerate}
	
	
\end{document}